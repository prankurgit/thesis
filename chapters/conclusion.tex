\section{PUF Toolkit}
In this work, an already developed software based toolkit named \emph{PUF Toolkit} was improved upon and extended. The implementation boosts the easy-to-use console user interface and provides the implementation of new common and proven metrics for the evaluation of the PUF responses. To be specific the following metrics were added to the toolkit: \emph{Jaccardi Index, Intra-Jaccardi Index, Inter-Jaccardi Index} and \emph{Hamming Distance}. The implementation of the metrics favored an object
oriented approach and hence the toolkit was designed in C++ and to ensure an effective and intuitive working with the toolkit the user interface was derived from the nine main design guidelines by Nielsen and Molich[seb 45]. To support the designers and reseachers in the evaluation of the PUF responses, the visualization of the metrics results was optimized with respect to PUF responses, offsets functionality was enhanced to also skip bytes from the end of the response and menu items were rearranged for better navigation. Apart from
these optimizations the visual layout of the result file was improved to contain new parameters like Offset from the end of the file, fractional Jaccardi Index and Fractional Hamming Distance. A consistent developement cycle and verification phase was performed for each metric implementation. The evaluation explained the interpretation of the results for the extended metrics and summarized already implemented metrics in a table. The user interface design guidelines were outlined
and resource-friendly working of the toolkit was presented and finally the testing of the toolkit was done with a large set of samples and the data was recorded and visualized as bar graphs.\\

\section{Extension and Integration of Fuzzy Extractor}
The already implemented BCH code based fuzzy extractor was added and integrated to the PUF Toolkit. The implementation of the fuzzy extractor was realized in C++ and it achieves a PUF-based secure key storage on hardware devices with limited resources. The integration part covered the combining of the data structures of the already implemented BCH code (based on Morelos-Zaragozas Encoder/Decoder for binary BCH codes in C [seb 43]) to the structure of the PUF Toolkit. Intensive code review was
done to correct some errors in the original Morelos-Zaragozas implementation to ensure that the encoder and decoder functionalities work even with large values of paramter $m$. The erroneous state recovery and error correction capability and other functionalities of the PUF Toolkit were merged to the fuzzy extractor. The PUF BCH Encoder represents the enrollment phase of the fuzzy extractor for the realization of PUF-based secure storage. The BCH code is highly configurable and the code
paramters (n, k, d) alongwith error correcting capability $t$ and parameter $m$ can be adjusted according to user needs and PUF properties of the desired PUF instance. The BCH codes create code words which are processed by a linear repetition code before the result is ``one time encrypted'' with an XOR operation with the PUF response. The Encoder menu adheres to the above mentioned design guidelines and provides an easy-to-use and well-structured user interface. The PUF BCH decoder
implementation is the counterpart of the PUF BCH Encoder and represents the second (reconstruction) phase of the fuzzy extractor. The reconstruction phase of the fuzzy extractor combines XOR operation with the PUF response and subsequently executes Majority Voting algorithm followed by the BCH decoder functionality to retrieve the original secret data.The integration procedure of the BCH decoder to the toolkit was similar to the integration procedure of the BCH encoder to the toolkit. The sub menu of the
decoder is intuitive and a well-structured user interface that is inspired from the BCH encoder menu. The factors of the linear repetition code and majority vote can take the value 7 or 15. For evaluation, the BCH fuzzy extractor was tested with different code parameters and the time requriements for the pre-processing step that creates a timing overhead were recoreded and visualized as a time chart. The recovered file was cross checked with the original secret to make sure they were
identical.\\

The toolkit was extended with the Golay based Fuzzy extractor. The original implementation of the Golay code was in very basic format which hardcoded the input file, PUF response and other data structures, and had very restrained functionality with no support for offsets, error recovery and user files input. All these functionalites were first added to the base version and then both the encoder and decoder parts of the Golay based Fuzzy extractor were integrated to the PUF Toolkit. The
golay encoder forms the generation phase of the fuzzy extractor. Similar to BCH encoder it performs XOR operation of the secret key with the PUF response and then applies linear repetition algorithm followed by golay encoding to generate the helper data that is used in the decoding phase to recover the encoded secret key. But unlike BCH code, the parameters of the Golay code are fixed to (23, 12, 7), due to this limitation the recovery of large files that resulted in a helper data bigger than
the PUF response is not possible and the user is suggested to use BCH based fuzzy extractor instead. The encoder also appends the configuration settings like offsets, input filesize and linear repetition factor to the helper data. The Golay decoder constitutes the reconstruction phase of the fuzzy extractor. It reads the saved configuration settings from the end of the helper file and then proceeds to a XOR operation of the helper
data with the PUF response followed by the Majority Voting procedure with the same factor as the one used in by the linear repetition algorithm in the encoder part. Lastly based on the settings and code parameters (n, k, d) the decoder creates a decoding table and restores the original secret from the helper data, thus completing the PUF-based secure storage recovery. The menus for both encoder and decoder are derived from the BCH based fuzzy extractor and
adhere to the same design guidelines as the PUF Toolkit, thereby making its user interface easy-to-understand and well-structured. The evaluation of the Golay code was done by testing the encoder and decoder for their time requirements and CPU usage with different PUF response sizes by adjusting the offsets and represnted as bar graph for analysis. A final check was performed using the ``diff'' utility to ensure that the recovered file does match the original input secret key.

\section{Integration to the CogniCrypt}
The metrics of the toolkit were made available to the Java developers for evaluation of PUF responses via the opensource library CogniCrypt. Using this library the developers with no prior knowledge about the PUF Toolkit can develop programs in Java by calling the functions of the toolkit. This was made possible by developing a seperate module to interface the C++ code with the Java runtime via the Oracle Standard technology Java Native Interface. JNI is a programming
framework that enables the Java virtual machine to call the functions of the native applications and libraries written in other languages like C, C++ and assemby. The JNI enabled toolkit is compiled as a shared library and packaged as a JAR (Java Archive) to be included in the Java project as external library for developement. The CogniCrypt is an opensource library [cogni.pdf] based on Clafer lightweight modelling language [clafer.pdf] and was developed to assist Java developers for secure and correct usage of the cryptographic APIs. It asks
simple questions from the developers to select the appropriate algorithm based on the answers and generate a boilerplate code that shows the sample usage of the cryptographic API. For this thesis the specific metrics of the PUF Toolkit namely: \emph{Hamming Weight, Intra-Hamming Distance, Inter-Hamming Distance, (Shannon) Entropy, min-Entropy} and \emph{median and average} were implemented as clafer algorithms and added to the CogniCrypt model. The graphical user interface of the
CogniCrypt is based on the advices taken from a survey, done with Java developers as audience, on how to improve the usability of cryptographical APIs. Thus making it simple, intuitive and easy to use.\\

The extended and improved \emph{PUF Toolkit} integrated with the \emph{Fuzzy Extractor} functionality and interfaced to the \emph{CogniCrypt} opensource library, contributes to and accelerate the developement of new security approaches. On one hand it is helpful in the evaluation of PUFs in the context of security applications and on the other hand it provides its functionalities to be used in a Java project and can be further used to realize new or enhanced security techniques that
are based on PUFs. Although there might be additional aspects that could be optimized in the future to further increase the PUF Toolkit application area, this thesis enhances the scope of the initial \emph{PUF Toolkit} to assists the reseachers and developers in the evaluation of PUF instances.


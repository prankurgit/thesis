As the internet grows so does the threat to its infrastructure.
In recent wake of the Heartbleed bug, the threat to cloud storage and the NSA debacle tracking private data, security becomes more important than ever.\\

Well-known protection mechanisms like symmetric shared key-based or asymmetric public/private key-based schemes require algorithms to generate secure keys and/or pass certificates between two communicating parties. Generation and storage of keys and certificates on embedded devices with limited hardware resources is cost ineffective. Securing the data in the memory requires specialized hardware security modules which are expensive. For example, a specialized tamper-proof hardware may cost more than 3000 USD
\cite{os8}.\\

Physical(ly) Unclonable Functions (PUFs) \cite{os9} provide a low-cost solution to the economic problem of key-generation on limited resource hardware embedded devices. PUFs are a result of the manufacturing variations while printing the Integrated Circuits (ICs); these variations are unclonable which means even the manufacturer cannot generate two identical ICs. A PUF depends on its physical microstructure and exhibits challenge/response behavior to evaluate its structure. It is a hardware analogy to a one-way function that is easy to evaluate but difficult to predict. When a physical stimulus (challenge) is applied to the structure it outputs an unpredictable (but repeatable) response which depends on the physical factors introduced during manufacturing of the PUF. So the PUF is unique and can be seen as a fingerprint of the hardware\cite{15}.\\

A fuzzy extractor \cite{fuzzy} can be applied to the PUF to derive an exclusive and strong cryptographic key from the underlying physical microstructure. Since, the output from the evaluation is reproducible and unique, every time the same key is generated. Some PUFs like SRAM PUFs may not need additional costs and are generated implicitly by variations in the manufacturing of SRAM modules itself.\\

There are a lot of applications of PUFs like realization of a secure key storage \cite{2,3,4}, Intellectual Property (IP) protection and remote service activation \cite{5,11}, anti-counterfeiting \cite{6,7}, device authentication and secret key generation \cite{8,9,10}, remote attestation protocols \cite{12,13} and the integration in cryptographic algorithms \cite{14}. Based on the use case, PUF instances must satisfy different criteria and implement special properties to be integrated
into the security mechanism.\\

The behavior of a PUF instance is sensitive to differing operational conditions such as supply voltage, ambient temperature or aging effects that can result in slightly different PUF responses each time the PUF is challenged. Consequently an adequate pre-assessment of the PUF-instance is required to ensure that the response is consistent, required criteria are accomplished and essential properties are provided for the desired security mechanism. This pre-evaluation also verifies the stability of
the PUF reaction under fluctuating operating conditions. For security mechanism, correct and stable working behavior plays a major role; specifically if the mechanism was built on top of a noisy PUF-instance or if the PUF-instance is the central building block.\\

CogniCrypt is an opensource library that is implemented as an Eclipse plugin that supports Java developers in using Java Cryptographic APIs \cite{cogni}. Cryptographical algorithms are highly configurable and a configuration which may be secure in one scenario might have flaws and loopholes in another. CogniCrypt aims to provide non-security expert Java developers an opportunity to securely and correctly use the underlying cryptographical APIs.

It classifies the different configurations based on developer's answers to high-level questions in non-expert terminology \cite{onward2015}. Apart from generating secure implementations for cryptography tasks it also analyzes developer code and generates alerts for misuses of cryptographic APIs \cite{cogni}. 


\section{Goal of the Thesis}
The goal is to extend the user interface that helps designers and researchers to evaluate PUF responses  and then integrate the toolkit as a JAR archive in CogniCrypt wrapped in Java Native Interface.

This Thesis is an extension of the Master Thesis ``Development of a user interface and implementation of specific software tools for the evaluation and realization of PUFs with respect to security applications''\cite{71} where the original UI was designed and implemented. The design did not incorporate BCH encoding and decoding in the toolkit and was presented as separate UI. The secure key storage approach in \cite{10} was realized using a fuzzy extractor, which was based on the syndrome
generation of a Golay(23,12,7) error-correction code in combination with a binary linear repetition code and a binary majority voting for the reconstruction\cite{71}. Golay (23, 12, 7) error-correction code can correct three errors in each syndrome. Therefore an implementation of this error-correction code was used to realize the fuzzy extractor, which was included in the toolkit.

\section{Contribution}
This Thesis aims to improve the UI, add new functionalities to the PUF response evaluation, integrate a BCH and a Golay encoder and decoder for fuzzy extractor scheme to the toolkit.
Then wrap the toolkit in Jar to be accessible by CogniCrypt via Java Native Interface.

In particular, the contribution of this work is characterized by the following points:

\begin{itemize}
	\item Improvement of a console user interface for the PUF Toolkit in C++.
	\item Implementation of specific software tools with respect to PUF relevant metrics in C++.
		\begin{itemize}
			\item Hamming Distance Menu
			\item Golay encoder 
			\item Golay Decoder
			\item Intra Jaccard Index
			\item Inter Jaccard Index
		\end{itemize}
	\item BCH encoder and decoder integration to the PUF Toolkit.
	\item Augmenting the offset functionality.
	\item Enhancing the toolkit to output fractional distance.
	\item Wrapping the toolkit via JNI to be made accessible in CogniCrypt.
	\item UI implementation for CogniCrypt module to generate boilerplate code.
\end{itemize}

\section{Outline}
The present work is divided into the following chapters: Chapter 2 covers the state of the art research with a focus on Physical(ly) Unclonable Functions, their definition, classification, working principles, and categorization.In addition, some information and mathematical concepts behind linear codes and cyclic codes like BCH error-correction code are given. In Chapter 3, the original version of the PUF Toolkit is summarized and extended metrics implementation and improvement
to the Toolkit is explained. The integration of the fuzzy extractor to the toolkit and extension of the fuzzy extractor is present thereafter. The integration of the PUF Toolkit into CogniCrypt, an Opensource Java-based library is presented towards the end of Chapter 3. The testing and evaluation of the Toolkit are presented in Chapter 4. A final conclusion can be found in Chapter 5 and future work is proposed in Chapter 6.
